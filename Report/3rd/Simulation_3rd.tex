\documentclass[11pt,titlepage]{jsarticle}
\usepackage{listings}
\usepackage{jlisting}
\usepackage[T1]{fontenc}
\usepackage{textcomp}
\usepackage{ascmac}
\usepackage[dvipdfmx]{graphicx}
\usepackage{array}
\usepackage{float}
\usepackage{moreverb}
\usepackage{framed}

\renewcommand{\lstlistingname}{ソースコード}
\lstset{
  language=c,
  basicstyle=\ttfamily\scriptsize,
  commentstyle=\textit,
  classoffset=1,
  keywordstyle=\bfseries,
  frame=tRBl,
  framesep=5pt,
  showstringspaces=false,
  numbers=left,
  stepnumber=1,
  numberstyle=\tiny,
  tabsize=2,
  breaklines=true
}
\setcounter{page}{1}

\title{シミュレーション}
\author{4年電子情報工学科\\34番 横前洸佑}
\date{提出日:2020/2/17(月)\\	提出期限:2020/2/17(月)10:00}
\begin{document}

\maketitle

\section{課題11}
課題11では、ガウスの消去法を用いて連立方程式(\ref{eq:gauss})を解き、$x_1, x_2, x_3, x_4$を求める。


\begin{eqnarray}
\label{eq:gauss}
		\begin{array}{l}
			x_1 + 2x_2 + x_3 + 5x_4 = 20.5\\
			8x_1 + x_2 + 3x_3 + x_4 = 14.5\\
			x_1 + 7x_2 + x_3 + x_4 = 18.5\\
			x_1 + x_2 + 6x_3 + x_4 = 9.0\\
		\end{array}
\end{eqnarray}

\subsection{作成したプログラム}
今回作成したプログラムをソースコード\ref{src:kadai11}に示す。

\lstinputlisting[caption=課題11のプログラム,label=src:kadai11]{src/kadai11.c}
ガウスの消去法の計算部分を関数として用意し、main関数内から呼び出している。

\subsection{プログラムの実行結果}
実行結果を以下に示す。
\begin{oframed}
\verbatimtabinput[2]{result/kadai11.txt}
\end{oframed}

\subsection{考察}
元の方程式に計算によって得た値を代入する。
\begin{eqnarray}
\label{eq:gauss_result}
		\begin{array}{l}
			1.0 + 2\times2.0 + 0.5 + 5\times3.0 = 20.5\\
			8\times1.0 + 2.0 + 3\times0.5 + 3.0 = 14.5\\
			1.0 + 7\times2.0 + 0.5 + 3.0 = 18.5\\
			1.0 + 2.0 + 6\times0.5 + 3.0 = 9.0\\
		\end{array}
\end{eqnarray}
式がすべて成り立つ。
この結果よりプログラムは正しく動作していると言える。

\section{課題12}
課題12では、ガウスの消去法を用いて連立方程式(\ref{eq:gauss_pivot})を解き、$x_1, x_2, x_3, x_4$を求める。
また、ピボット選択なしで、float型で変数を宣⾔した場合とdouble型で変数を宣⾔した場合について解いた時、ピボット選択ありで、float型で変数を宣⾔した場合とdouble型で変数を宣⾔した場合について解いた時の結果の違いを考察する。


\begin{eqnarray}
\label{eq:gauss_pivot}
		\begin{array}{l}
			1.0x_1 + 0.96x_2 + 0.84x_3 + 0.64x_4 = 3.44\\
			0.96x_1 + 0.9214x_2 + 0.4406x_3 + 0.2222x_4 = 2.5442\\
			0.84x_1 + 0.4406x_2 + 1.0x_3 + 0.3444x_4 = 2.6250\\
			0.64x_1 + 0.2222x_2 + 0.3444x_3 + 1.0x_4 = 2.2066\\
		\end{array}
\end{eqnarray}

\subsection{作成したプログラム}
今回作成したプログラムをソースコード\ref{src:kadai12}に示す。

\lstinputlisting[caption=課題12のプログラム,label=src:kadai12]{src/kadai12.c}
ガウスの消去法の計算部分を関数として用意し、main関数内から呼び出している。
また、{\tt gauss()}関数内36行目でピボット選択を行う。

\subsection{プログラムの実行結果}
ピボット選択なしで、float型で実行した結果を以下に示す。

\begin{oframed}
\verbatimtabinput[2]{result/kadai12_nonpivot_float.txt}
\end{oframed}

ピボット選択なしで、double型で実行した結果を以下に示す。

\begin{oframed}
\verbatimtabinput[2]{result/kadai12_nonpivot_double.txt}
\end{oframed}

ピボット選択ありで、float型で実行した結果を以下に示す。

\begin{oframed}
\verbatimtabinput[2]{result/kadai12_nonpivot_float.txt}
\end{oframed}

ピボット選択ありで、double型で実行した結果を以下に示す。

\begin{oframed}
\verbatimtabinput[2]{result/kadai12_pivot_double.txt}
\end{oframed}

\subsection{考察}
元の方程式に計算によって得た値を代入する。
\begin{eqnarray}
\label{eq:gauss_pivot_result}
		\begin{array}{l}
			1.0 + 0.96 + 0.84 + 0.64 = 3.44\\
			0.96 + 0.9214 + 0.4406 + 0.2222 = 2.5442\\
			0.84 + 0.4406 + 1.0 + 0.3444 = 2.6250\\
			0.64 + 0.2222 + 0.3444 + 1.0 = 2.2066\\
		\end{array}
\end{eqnarray}
式がすべて成り立つ。
この結果よりプログラムは正しく動作していると言える。

\section{課題13}
表\ref{table:kadai13}に示す7組のデータに対して2次式で近似を行うプログラムを作成する。


\begin{table}[H]
\caption{データ}
\label{table:kadai13}
\centering
\begin{tabular}{|c|c|c|c|c|c|c|c|}\hline
	$i$&1&2&3&4&5&6&7 \\ \hline
	$x_i$&0.0&0.1&0.2&0.3&0.4&0.5&0.6 \\ \hline
	$y_i$&0.000&0.034&0.138&0.282&0.479&0.724&1.120\\ \hline
\end{tabular}
\end{table}


\subsection{作成したプログラム}
今回作成したプログラムをソースコード\ref{src:kadai13}に示す。

\lstinputlisting[caption=課題13のプログラム,label=src:kadai13]{src/kadai13.c}

\subsection{プログラムの実行結果}
プログラムの実行結果を以下に示す。
\begin{oframed}
\verbatimtabinput[2]{result/kadai13.txt}
\end{oframed}

\section{課題14}



\end{document}